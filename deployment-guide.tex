\documentclass[12pt,a4paper]{article}
\usepackage[utf8]{inputenc}
\usepackage[polish]{babel}
\usepackage{geometry}
\usepackage{hyperref}
\usepackage{graphicx}
\usepackage{listings}
\usepackage{xcolor}
\usepackage{enumitem}

\geometry{margin=2.5cm}

\definecolor{codegreen}{rgb}{0,0.6,0}
\definecolor{codegray}{rgb}{0.5,0.5,0.5}
\definecolor{codepurple}{rgb}{0.58,0,0.82}
\definecolor{backcolour}{rgb}{0.95,0.95,0.92}

\lstdefinestyle{mystyle}{
    backgroundcolor=\color{backcolour},   
    commentstyle=\color{codegreen},
    keywordstyle=\color{magenta},
    numberstyle=\tiny\color{codegray},
    stringstyle=\color{codepurple},
    basicstyle=\ttfamily\footnotesize,
    breakatwhitespace=false,         
    breaklines=true,                 
    captionpos=b,                    
    keepspaces=true,                 
    numbers=left,                    
    numbersep=5pt,                  
    showspaces=false,                
    showstringspaces=false,
    showtabs=false,                  
    tabsize=2
}

\lstset{style=mystyle}

\title{\textbf{Instrukcja Wdrożenia Kursu Online}\\
\large Projekt-M Demo - Trygonometria}
\author{Kurs Trygonometrii - Deployment Guide}
\date{\today}

\begin{document}

\maketitle

\tableofcontents
\newpage

\section{Przygotowanie do wdrożenia}

\subsection{Wymagania wstępne}
Przed rozpoczęciem procesu wdrożenia upewnij się, że masz:
\begin{itemize}
    \item Zainstalowany Node.js (wersja 14 lub nowsza)
    \item Dostęp do terminala/konsoli
    \item Projekt znajdujący się w folderze \texttt{projekt-m-demo}
    \item Połączenie z internetem
\end{itemize}

\subsection{Weryfikacja instalacji}
Otwórz terminal i wykonaj:
\begin{lstlisting}[language=bash]
node --version
npm --version
\end{lstlisting}

\section{Budowanie aplikacji}

\subsection{Krok 1: Przejście do folderu projektu}
\begin{lstlisting}[language=bash]
cd /Users/krzysztofkilian/Desktop/projekt-m-demo
\end{lstlisting}

\subsection{Krok 2: Instalacja zależności (jeśli nie wykonano wcześniej)}
\begin{lstlisting}[language=bash]
npm install
\end{lstlisting}

\subsection{Krok 3: Budowanie produkcyjnej wersji}
\begin{lstlisting}[language=bash]
npm run build
\end{lstlisting}

\textbf{Ważne:} Ten proces może potrwać 1-3 minuty. Po zakończeniu zobaczysz komunikat:
\begin{verbatim}
The build folder is ready to be deployed.
\end{verbatim}

\subsection{Krok 4: Weryfikacja}
Sprawdź, czy folder \texttt{build} został utworzony:
\begin{lstlisting}[language=bash]
ls -la build/
\end{lstlisting}

\newpage
\section{Wdrożenie na Vercel}

\subsection{Metoda 1: Przez przeglądarkę (najprostsza)}

\subsubsection{Krok 1: Rejestracja}
\begin{enumerate}
    \item Wejdź na stronę: \url{https://vercel.com}
    \item Kliknij \textbf{Sign Up}
    \item Zaloguj się przez GitHub, GitLab lub email
\end{enumerate}

\subsubsection{Krok 2: Upload projektu}
\begin{enumerate}
    \item Po zalogowaniu kliknij \textbf{Add New...} → \textbf{Project}
    \item Wybierz opcję \textbf{Deploy from Local}
    \item Przeciągnij folder \texttt{build} na stronę
    \item Poczekaj na upload (zwykle 10-30 sekund)
\end{enumerate}

\subsubsection{Krok 3: Konfiguracja}
\begin{enumerate}
    \item Nazwa projektu: \texttt{trigonometry-course} (lub dowolna)
    \item Framework Preset: wybierz \textbf{Other}
    \item Build Command: zostaw puste
    \item Output Directory: zostaw puste lub wpisz \texttt{.}
    \item Kliknij \textbf{Deploy}
\end{enumerate}

\subsubsection{Krok 4: Gotowe}
\begin{enumerate}
    \item Po około 30 sekundach otrzymasz link typu:
    \begin{verbatim}
    https://trigonometry-course.vercel.app
    \end{verbatim}
    \item Kliknij link aby zobaczyć działającą aplikację
\end{enumerate}

\subsection{Metoda 2: Przez CLI (dla zaawansowanych)}

\subsubsection{Instalacja Vercel CLI}
\begin{lstlisting}[language=bash]
npm install -g vercel
\end{lstlisting}

\subsubsection{Deploy}
\begin{lstlisting}[language=bash]
cd /Users/krzysztofkilian/Desktop/projekt-m-demo
vercel --prod ./build
\end{lstlisting}

Postępuj zgodnie z promptami:
\begin{itemize}
    \item Set up and deploy? \textbf{Y}
    \item Which scope? Wybierz swoje konto
    \item Link to existing project? \textbf{N}
    \item Project name? \texttt{trigonometry-course}
    \item Directory? \texttt{./build}
    \item Override settings? \textbf{N}
\end{itemize}

\newpage
\section{Wdrożenie na Netlify}

\subsection{Metoda 1: Drag and Drop}

\subsubsection{Krok 1: Przygotowanie}
\begin{enumerate}
    \item Wejdź na: \url{https://app.netlify.com/drop}
    \item Nie musisz się logować (ale możesz dla zachowania projektu)
\end{enumerate}

\subsubsection{Krok 2: Upload}
\begin{enumerate}
    \item Otwórz Finder/Explorer
    \item Znajdź folder: \texttt{projekt-m-demo/build}
    \item Przeciągnij cały folder \texttt{build} na stronę Netlify
    \item Poczekaj na upload (10-30 sekund)
\end{enumerate}

\subsubsection{Krok 3: Gotowe}
\begin{enumerate}
    \item Otrzymasz link typu:
    \begin{verbatim}
    https://amazing-curie-521a4b.netlify.app
    \end{verbatim}
    \item Możesz zmienić nazwę domeny w ustawieniach
\end{enumerate}

\subsection{Metoda 2: Git Integration}

\subsubsection{Krok 1: Push do GitHub}
\begin{lstlisting}[language=bash]
cd /Users/krzysztofkilian/Desktop/projekt-m-demo
git init
git add .
git commit -m "Initial commit"
git remote add origin https://github.com/twoje-konto/trigonometry-course.git
git push -u origin main
\end{lstlisting}

\subsubsection{Krok 2: Połączenie z Netlify}
\begin{enumerate}
    \item Zaloguj się na \url{https://netlify.com}
    \item Kliknij \textbf{Add new site} → \textbf{Import an existing project}
    \item Wybierz GitHub i autoryzuj
    \item Wybierz repozytorium
    \item Ustawienia budowania:
    \begin{itemize}
        \item Build command: \texttt{npm run build}
        \item Publish directory: \texttt{build}
    \end{itemize}
    \item Kliknij \textbf{Deploy site}
\end{enumerate}

\newpage
\section{Wdrożenie na GitHub Pages}

\subsection{Przygotowanie}

\subsubsection{Krok 1: Modyfikacja package.json}
Dodaj do pliku \texttt{package.json}:
\begin{lstlisting}[language=json]
"homepage": "https://twoje-konto.github.io/trigonometry-course"
\end{lstlisting}

\subsubsection{Krok 2: Instalacja gh-pages}
\begin{lstlisting}[language=bash]
npm install --save-dev gh-pages
\end{lstlisting}

\subsubsection{Krok 3: Dodanie skryptów}
W \texttt{package.json} w sekcji \texttt{scripts} dodaj:
\begin{lstlisting}[language=json]
"predeploy": "npm run build",
"deploy": "gh-pages -d build"
\end{lstlisting}

\subsection{Deployment}

\subsubsection{Krok 1: Utworzenie repozytorium}
\begin{lstlisting}[language=bash]
cd /Users/krzysztofkilian/Desktop/projekt-m-demo
git init
git add .
git commit -m "Initial commit"
git remote add origin https://github.com/twoje-konto/trigonometry-course.git
git push -u origin main
\end{lstlisting}

\subsubsection{Krok 2: Deploy}
\begin{lstlisting}[language=bash]
npm run deploy
\end{lstlisting}

\subsubsection{Krok 3: Aktywacja GitHub Pages}
\begin{enumerate}
    \item Wejdź na GitHub do swojego repozytorium
    \item Settings → Pages
    \item Source: Deploy from a branch
    \item Branch: \texttt{gh-pages}
    \item Folder: \texttt{/ (root)}
    \item Save
\end{enumerate}

\subsubsection{Krok 4: Dostęp}
Po około 5 minutach aplikacja będzie dostępna pod:
\begin{verbatim}
https://twoje-konto.github.io/trigonometry-course
\end{verbatim}

\newpage
\section{Porównanie platform}

\begin{table}[h]
\centering
\begin{tabular}{|l|c|c|c|}
\hline
\textbf{Cecha} & \textbf{Vercel} & \textbf{Netlify} & \textbf{GitHub Pages} \\
\hline
Darmowy limit & 100 GB & 100 GB & 100 GB \\
\hline
Własna domena & Tak & Tak & Tak \\
\hline
HTTPS & Automatycznie & Automatycznie & Automatycznie \\
\hline
Deploy czas & 30 sekund & 30 sekund & 5 minut \\
\hline
Łatwość & Bardzo łatwe & Bardzo łatwe & Średnie \\
\hline
CI/CD & Tak & Tak & Tak (Actions) \\
\hline
\end{tabular}
\caption{Porównanie platform hostingowych}
\end{table}

\section{Rozwiązywanie problemów}

\subsection{Problem: Build failed}
\textbf{Rozwiązanie:}
\begin{lstlisting}[language=bash]
rm -rf node_modules package-lock.json
npm install
npm run build
\end{lstlisting}

\subsection{Problem: Blank page po deploy}
\textbf{Rozwiązanie:}
\begin{enumerate}
    \item Sprawdź konsolę przeglądarki (F12)
    \item Upewnij się, że w \texttt{package.json} nie ma \texttt{homepage} (chyba że GitHub Pages)
    \item Sprawdź czy wszystkie pliki są w folderze \texttt{build}
\end{enumerate}

\subsection{Problem: 404 na Netlify/Vercel}
\textbf{Rozwiązanie:}
Utwórz plik \texttt{public/\_redirects} (Netlify) lub \texttt{vercel.json} (Vercel):

Dla Netlify (\texttt{public/\_redirects}):
\begin{lstlisting}
/*    /index.html   200
\end{lstlisting}

Dla Vercel (\texttt{vercel.json}):
\begin{lstlisting}[language=json]
{
  "rewrites": [{ "source": "/(.*)", "destination": "/" }]
}
\end{lstlisting}

\section{Aktualizacja aplikacji}

\subsection{Lokalne zmiany}
\begin{lstlisting}[language=bash]
# Edytuj pliki w src/
# Następnie:
npm run build
\end{lstlisting}

\subsection{Deploy aktualizacji}

\subsubsection{Vercel/Netlify z Git}
\begin{lstlisting}[language=bash]
git add .
git commit -m "Update: opis zmian"
git push
# Automatyczny deploy w 1-2 minuty
\end{lstlisting}

\subsubsection{Vercel/Netlify manual}
Powtórz proces drag \& drop z nowym folderem \texttt{build}

\subsubsection{GitHub Pages}
\begin{lstlisting}[language=bash]
npm run deploy
\end{lstlisting}

\section{Koszty i limity}

\subsection{Darmowe limity}
\begin{itemize}
    \item \textbf{Vercel:} 100 GB transferu/miesiąc, nieograniczone projekty
    \item \textbf{Netlify:} 100 GB transferu/miesiąc, 300 minut budowania
    \item \textbf{GitHub Pages:} 100 GB transferu/miesiąc, 10 budowań/godzinę
\end{itemize}

\subsection{Szacowane użycie}
Dla kursu z 194 zadaniami:
\begin{itemize}
    \item Rozmiar aplikacji: ~2 MB
    \item 1000 użytkowników/miesiąc = ~2 GB transferu
    \item 10000 użytkowników/miesiąc = ~20 GB transferu
    \item \textbf{Wniosek:} Darmowy plan wystarczy dla 50000 użytkowników/miesiąc
\end{itemize}

\section{Własna domena (opcjonalnie)}

\subsection{Krok 1: Kup domenę}
Popularne serwisy:
\begin{itemize}
    \item Namecheap: od \$8/rok
    \item Google Domains: od \$12/rok
    \item OVH: od 20 zł/rok (.pl)
\end{itemize}

\subsection{Krok 2: Konfiguracja DNS}

\subsubsection{Dla Vercel}
Dodaj rekord CNAME:
\begin{verbatim}
Type: CNAME
Name: @
Value: cname.vercel-dns.com
\end{verbatim}

\subsubsection{Dla Netlify}
Dodaj rekord CNAME:
\begin{verbatim}
Type: CNAME
Name: @
Value: twoja-aplikacja.netlify.app
\end{verbatim}

\subsection{Krok 3: Dodanie domeny w panelu}
\begin{enumerate}
    \item Wejdź do panelu Vercel/Netlify
    \item Settings → Domains
    \item Add domain
    \item Wpisz swoją domenę
    \item Postępuj zgodnie z instrukcjami
\end{enumerate}

\section{Podsumowanie}

\subsection{Rekomendacje}
\begin{enumerate}
    \item \textbf{Dla początkujących:} Netlify (drag \& drop)
    \item \textbf{Dla szybkiego deploy:} Vercel
    \item \textbf{Dla integracji z GitHub:} GitHub Pages
    \item \textbf{Dla produkcji:} Vercel lub Netlify z Git
\end{enumerate}

\subsection{Czas całkowity}
\begin{itemize}
    \item Build aplikacji: 2-3 minuty
    \item Upload i konfiguracja: 2-5 minut
    \item Propagacja DNS (własna domena): do 48 godzin
    \item \textbf{Razem:} 5-10 minut do działającej aplikacji
\end{itemize}

\subsection{Wsparcie}
W razie problemów:
\begin{itemize}
    \item Vercel: \url{https://vercel.com/docs}
    \item Netlify: \url{https://docs.netlify.com}
    \item GitHub Pages: \url{https://docs.github.com/pages}
\end{itemize}

\end{document}