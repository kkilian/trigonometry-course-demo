\documentclass[12pt,a4paper]{article}
\usepackage[utf8]{inputenc}
\usepackage[polish]{babel}
\usepackage{amsmath}
\usepackage{geometry}
\usepackage{array}
\usepackage{booktabs}
\usepackage{longtable}
\geometry{margin=2.5cm}

\title{Metodologia generowania zadań maturalnych\\
\large Pełne pokrycie wymagań CKE}
\author{}
\date{}

\begin{document}

\maketitle

\section{Proces zapewnienia pełnego pokrycia wymagań}

\subsection{Krok 1: Analiza wymagań egzaminacyjnych CKE}

Dla każdego działu należy zidentyfikować wymagania szczegółowe z podstawy programowej:

\subsubsection{Poziom podstawowy}
\begin{itemize}
    \item P1 -- Wymagania podstawowe (definicje, proste operacje)
    \item P2 -- Rozwiązywanie równań podstawowych
    \item P3 -- Rozwiązywanie nierówności podstawowych
    \item P4 -- Interpretacja i zastosowanie
\end{itemize}

\subsubsection{Poziom rozszerzony}
\begin{itemize}
    \item R1 -- Równania złożone
    \item R2 -- Nierówności złożone i metody zaawansowane
    \item R3 -- Zadania z parametrem
    \item R4 -- Łączenie z innymi działami
\end{itemize}

\subsection{Krok 2: Tworzenie matrycy pokrycia}

Matryca pokrycia to tabela krzyżowa gdzie:
\begin{itemize}
    \item \textbf{Wiersze}: konkretne wymagania z podstawy programowej
    \item \textbf{Kolumny}: wygenerowane zadania
    \item \textbf{Komórki}: oznaczenie które zadanie sprawdza które wymaganie
\end{itemize}

\begin{table}[h]
\centering
\begin{tabular}{|l|c|c|c|c|c|c|c|c|c|}
\hline
\textbf{Typ zadania} & \textbf{P1} & \textbf{P2} & \textbf{P3} & \textbf{P4} & \textbf{R1} & \textbf{R2} & \textbf{R3} & \textbf{R4} & \textbf{Liczba} \\
\hline
Podstawowe & $\checkmark$ & & & & & & & & 5 \\
Równania proste & $\checkmark$ & $\checkmark$ & & & & & & & 7 \\
Nierówności proste & $\checkmark$ & & $\checkmark$ & $\checkmark$ & & & & & 8 \\
Równania złożone & $\checkmark$ & & & & $\checkmark$ & & & & 8 \\
Metoda zaawansowana & $\checkmark$ & & & $\checkmark$ & & $\checkmark$ & & & 10 \\
Z parametrem & $\checkmark$ & & & & & & $\checkmark$ & & 7 \\
Zadania łączone & & & & & & & & $\checkmark$ & 5 \\
\hline
\textbf{SUMA} & & & & & & & & & \textbf{50} \\
\hline
\end{tabular}
\end{table}

\subsection{Krok 3: Określenie minimalnej liczby zadań}

\subsubsection{Zasada minimalnego pokrycia}
\begin{itemize}
    \item Każde wymaganie musi być pokryte przez \textbf{minimum 3 zadania}
    \item Zadania mogą pokrywać więcej niż jedno wymaganie
    \item Cel: minimalizacja liczby zadań przy zachowaniu pełnego pokrycia
\end{itemize}

\subsubsection{Rekomendowana liczba zadań}
\begin{itemize}
    \item \textbf{Poziom podstawowy}: 25--30 zadań
    \item \textbf{Poziom rozszerzony}: dodatkowo 20--25 zadań
    \item \textbf{RAZEM}: 45--55 zadań dla pełnego pokrycia
\end{itemize}

\subsection{Krok 4: Planowanie rozkładu trudności}

\begin{table}[h]
\centering
\begin{tabular}{|l|c|c|l|}
\hline
\textbf{Poziom} & \textbf{Liczba zadań} & \textbf{Procent} & \textbf{Charakterystyka} \\
\hline
Łatwy & 15 & 30\% & Jedno przekształcenie, oczywista dziedzina \\
Średni & 25 & 50\% & 2--3 przekształcenia, standardowe metody \\
Trudny & 10 & 20\% & Wieloetapowe, parametr, łączenie działów \\
\hline
\end{tabular}
\end{table}

\subsection{Krok 5: System walidacji pokrycia}

\subsubsection{Checklist weryfikacyjny}
\begin{itemize}
    \item[$\square$] Każde wymaganie pokryte min. 3 zadaniami
    \item[$\square$] Rozkład trudności: 30\% łatwe, 50\% średnie, 20\% trudne
    \item[$\square$] Min. 20\% zadań z kontekstem praktycznym
    \item[$\square$] Każdy typ błędu uczniowskiego zaadresowany
    \item[$\square$] Progresja trudności w ramach każdego typu
\end{itemize}

\subsubsection{Metryki pokrycia}
\begin{itemize}
    \item \textbf{Coverage ratio}: liczba pokrytych wymagań / wszystkie wymagania (cel: 100\%)
    \item \textbf{Redundancja}: średnia liczba zadań na wymaganie (cel: 3--5)
    \item \textbf{Balans}: odchylenie standardowe liczby zadań między wymaganiami (cel: $<2$)
\end{itemize}

\section{Schemat generowania zadań}

\subsection{Etap 1: Losowanie struktury}
\begin{enumerate}
    \item Wybór typu zadania (równanie/nierówność)
    \item Określenie złożoności wyrażenia (prosty/złożony ułamek)
    \item Decyzja o dodaniu parametru
\end{enumerate}

\subsection{Etap 2: Dobór współczynników}
\begin{enumerate}
    \item Unikanie skomplikowanych obliczeń (preferowane małe liczby całkowite)
    \item Zapewnienie sensownych rozwiązań (nie za dużo rozwiązań)
    \item Kontrola dziedziny (miejsca zerowe mianownika)
\end{enumerate}

\subsection{Etap 3: Weryfikacja rozwiązalności}
\begin{enumerate}
    \item Sprawdzenie czy zadanie ma rozwiązanie
    \item Weryfikacja poziomu trudności
    \item Test czasu rozwiązania (zgodny z normami CKE)
\end{enumerate}

\subsection{Etap 4: Generowanie dystraktorów}
\begin{enumerate}
    \item Typowe błędy (zapomnienie o dziedzinie)
    \item Błędy znaku przy mnożeniu nierówności
    \item Gubienie rozwiązań
    \item Błędne przekształcenia algebraiczne
\end{enumerate}

\subsection{Etap 5: Dodanie kontekstu (opcjonalne)}
\begin{enumerate}
    \item 30\% zadań z treścią praktyczną
    \item Konteksty: fizyka, ekonomia, geometria
    \item Zachowanie realizmu matematycznego
\end{enumerate}

\section{Przykład zastosowania dla działu ``Równania i nierówności wymierne''}

\subsection{Zidentyfikowane wymagania}
\begin{itemize}
    \item P1: Wyznaczanie dziedziny wyrażenia wymiernego
    \item P2: Rozwiązywanie równań wymiernych sprowadzalnych do liniowych
    \item P3: Rozwiązywanie prostych nierówności wymiernych
    \item P4: Interpretacja rozwiązań na osi liczbowej
    \item R1: Równania wymierne prowadzące do kwadratowych/wielomianowych
    \item R2: Stosowanie metody przedziałów dla złożonych nierówności
    \item R3: Rozwiązywanie równań i nierówności z parametrem
    \item R4: Łączenie równań wymiernych z innymi działami
\end{itemize}

\subsection{Rozkład 50 zadań}
\begin{longtable}{|l|l|c|}
\hline
\textbf{Blok} & \textbf{Opis} & \textbf{Liczba zadań} \\
\hline
\endhead
Dziedzina & Proste i złożone wyrażenia wymierne & 5 \\
Równania liniowe & Typ $\frac{ax+b}{cx+d} = k$ & 7 \\
Nierówności proste & Typ $\frac{1}{x-a} > 0$ & 8 \\
Równania kwadratowe & Sprowadzalne do $ax^2+bx+c=0$ & 8 \\
Metoda przedziałów & Iloczyn wielu czynników & 10 \\
Zadania z parametrem & Dyskusja rozwiązań & 7 \\
Zadania łączone & Z funkcją kwadratową, trygonometrią, ciągami & 5 \\
\hline
\textbf{SUMA} & & \textbf{50} \\
\hline
\end{longtable}

\subsection{Typowe błędy do zaadresowania}
\begin{enumerate}
    \item \textbf{Zapominanie o dziedzinie} -- minimum 5 zadań
    \item \textbf{Błędne mnożenie przez mianownik} -- minimum 3 zadania
    \item \textbf{Zmiana znaku nierówności} -- minimum 4 zadania
    \item \textbf{Błędy w metodzie przedziałów} -- cały blok 10 zadań
    \item \textbf{Gubienie rozwiązań} -- zadania z parametrem
\end{enumerate}

\section{Proces iteracyjnego uzupełniania}

\begin{enumerate}
    \item Wygeneruj pierwsze 30 zadań według planu
    \item Sprawdź matrycę pokrycia
    \item Zidentyfikuj luki w pokryciu wymagań
    \item Dogeneruj brakujące typy (15--25 zadań)
    \item Walidacja z arkuszami maturalnymi z ostatnich 5 lat
    \item Korekta i finalne dostosowanie
\end{enumerate}

\section{Podsumowanie}

Ten proces zapewnia:
\begin{itemize}
    \item \textbf{100\% pokrycie wymagań CKE}
    \item \textbf{Optymalną liczbę zadań} (bez nadmiarowej redundancji)
    \item \textbf{Zgodność z rzeczywistymi arkuszami maturalnymi}
    \item \textbf{Adresowanie typowych błędów uczniowskich}
    \item \textbf{Możliwość systematycznej weryfikacji pokrycia}
\end{itemize}

\end{document}