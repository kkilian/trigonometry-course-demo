\documentclass[10pt,a4paper]{article}
\usepackage[utf8]{inputenc}
\usepackage[polish]{babel}
\usepackage{amsmath}
\usepackage{amsfonts}
\usepackage{amssymb}
\usepackage{geometry}
\usepackage{multicol}
\usepackage{fancyhdr}

\geometry{margin=1.5cm}
\pagestyle{fancy}
\fancyhf{}
\fancyhead[C]{\textbf{Ściąga wzorów - Kombinatoryka}}
\fancyfoot[C]{\thepage}

\setlength{\parindent}{0pt}
\setlength{\parskip}{0.3em}

\begin{document}

\begin{multicols}{2}

\section*{1. Wzory kombinatoryczne}

\subsection*{Wariacje z powtórzeniami}
$$V_n^k = n^k$$
\textit{k-elementowe wariacje z powtórzeniami ze zbioru n-elementowego}

\textbf{Przykłady:}
\begin{itemize}
\item $8^5 = 32768$ (5 osób, 8 pięter)
\item $3^5 = 243$ (5 osób, 3 piętra)
\end{itemize}

\subsection*{Wariacje bez powtórzeń}
$$V_n^k = \frac{n!}{(n-k)!}$$

\textbf{Przykłady:}
\begin{itemize}
\item $V_8^5 = \frac{8!}{3!} = 8 \cdot 7 \cdot 6 \cdot 5 \cdot 4$
\item $V_6^4 = \frac{6!}{2!} = 6 \cdot 5 \cdot 4 \cdot 3$
\item $P(12,2) = \frac{12!}{10!} = 12 \times 11 = 132$
\end{itemize}

\subsection*{Kombinacje (współczynnik dwumianowy)}
$$C_n^k = \binom{n}{k} = \frac{n!}{k!(n-k)!}$$

\textbf{Przykłady:}
\begin{itemize}
\item $\binom{16}{2} = \frac{16 \cdot 15}{2} = 120$
\item $\binom{10}{5} = \frac{10!}{5! \cdot 5!} = 252$
\item $\binom{8}{3} = \frac{8 \cdot 7 \cdot 6}{3 \cdot 2 \cdot 1} = 56$
\end{itemize}

\subsection*{Współczynnik wielomianowy}
$$\binom{n}{n_1, n_2, \ldots, n_k} = \frac{n!}{n_1! \cdot n_2! \cdot \ldots \cdot n_k!}$$

\subsection*{Permutacje z powtórzeniami}
$$\frac{n!}{n_1! \cdot n_2! \cdot \ldots \cdot n_k!}$$
\textit{gdzie $n_i$ to liczby powtórzeń poszczególnych elementów}

\section*{2. Zasady kombinatoryczne}

\subsection*{Zasada mnożenia}
Dla niezależnych wyborów:
$$n_1 \times n_2 \times \ldots \times n_k$$

\textbf{Przykłady:}
\begin{itemize}
\item $4 \times 5 \times 5 \times 5$ (liczby czterocyfrowe)
\item $2 \times 1 \times 7^7$ (numery telefonu)
\end{itemize}

\subsection*{Zasada włączeń i wyłączeń}
$$|A \cup B| = |A| + |B| - |A \cap B|$$

Dla trzech zbiorów:
$$|A \cup B \cup C| = |A| + |B| + |C| - |A \cap B| - |A \cap C| - |B \cap C| + |A \cap B \cap C|$$

\subsection*{Zasada dopełnienia}
$$\text{Szukane} = \text{Wszystkie} - \text{Niechciane}$$

\section*{3. Funkcje specjalne}

\subsection*{Funkcje podłogi i sufitu}
\begin{itemize}
\item $\lceil x \rceil$ - sufit (zaokrąglenie w górę)
\item $\lfloor x \rfloor$ - podłoga (zaokrąglenie w dół)
\end{itemize}

\textbf{Przykłady:}
\begin{itemize}
\item $\lceil \frac{100}{6} \rceil \cdot 6 = 17 \cdot 6 = 102$
\item $\lfloor \frac{999}{6} \rfloor \cdot 6 = 166 \cdot 6 = 996$
\end{itemize}

\subsection*{Operacje modulo}
$n \bmod r$ - reszta z dzielenia $n$ przez $r$

\section*{4. Wzory na ciągi}

\subsection*{Suma liczb naturalnych}
$$\sum_{i=1}^{n} i = \frac{n(n+1)}{2}$$

\textbf{Przykład:} $\frac{9 \cdot 10}{2} = 45$ (suma od 1 do 9)

\subsection*{Ciągi arytmetyczne}
Liczba elementów: $\frac{a_n - a_1}{r} + 1$

gdzie $r$ to różnica ciągu

\textbf{Przykład:} $\frac{996 - 102}{6} + 1 = 150$

\section*{5. Lemat Burnside'a}

$$|X/G| = \frac{1}{|G|} \sum_{g \in G} |X^g|$$

gdzie:
\begin{itemize}
\item $X$ - zbiór obiektów
\item $G$ - grupa symetrii
\item $X^g$ - obiekty niezmiennicze względem $g$
\end{itemize}

\textbf{Przykład:} $|X/G| = \frac{1}{24}(1 \cdot 720 + 23 \cdot 0) = 30$

\section*{6. Wzory z teorii grafów}

\subsection*{Cykle Hamiltona}
$(n-1)!$ cykli w grafie pełnym na $n$ wierzchołkach

\textbf{Przykład:} $\frac{4!}{4} = 6$ cykli dla grafu $K_4$

\subsection*{Graf pełny}
$\binom{n}{2}$ krawędzi w grafie pełnym na $n$ wierzchołkach

\textbf{Przykład:} $\binom{5}{2} = 10$ krawędzi w $K_5$

\section*{7. Derangements (bez punktów stałych)}

$$D_n = n! \sum_{k=0}^n \frac{(-1)^k}{k!}$$

\section*{8. Symbole matematyczne}

\begin{itemize}
\item $!$ - silnia (factorial)
\item $n!$ - iloczyn liczb od 1 do $n$
\item $\binom{n}{k}$ - symbol Newtona
\item $\sum$ - suma (summation)
\item $\prod$ - iloczyn (product)
\item $\lceil \rceil$ - sufit (ceiling)
\item $\lfloor \rfloor$ - podłoga (floor)
\item $\bmod$ - modulo (reszta z dzielenia)
\end{itemize}

\section*{9. Kluczowe koncepty}

\textbf{Niezależność wyborów} - umożliwia zastosowanie zasady mnożenia

\textbf{Rozróżnialność obiektów} - wpływa na wybór między permutacjami a kombinacjami

\textbf{Symetria} - wymaga użycia lematu Burnside'a

\textbf{Ograniczenia} - prowadzą do zasady dopełnienia lub włączeń-wyłączeń

\section*{10. Przydatne wzory}

\subsection*{Małe silnie}
\begin{itemize}
\item $0! = 1$
\item $1! = 1$
\item $2! = 2$
\item $3! = 6$
\item $4! = 24$
\item $5! = 120$
\item $6! = 720$
\item $7! = 5040$
\item $8! = 40320$
\item $9! = 362880$
\item $10! = 3628800$
\end{itemize}

\subsection*{Małe potęgi}
\begin{itemize}
\item $2^{10} = 1024$
\item $3^5 = 243$
\item $4^3 = 64$
\item $5^3 = 125$
\item $8^3 = 512$
\item $9^2 = 81$
\item $10^3 = 1000$
\end{itemize}

\subsection*{Przydatne kombinacje}
\begin{itemize}
\item $\binom{n}{0} = 1$
\item $\binom{n}{1} = n$
\item $\binom{n}{n} = 1$
\item $\binom{n}{k} = \binom{n}{n-k}$
\end{itemize}

\end{multicols}

\end{document}