Na parterze wieżowca mającego osiem pięter do windy wsiada pięć osób. Na ile sposobów te osoby mogą wysiąść na piętrach, jeśli każda z tych osób wybiera piętro dowolnie? Pomijamy kolejność wysiadania z windy na poszczególnych piętrach.
Na parterze wieżowca mającego osiem pięter do windy wsiada pięć osób. Na ile sposobów te osoby mogą wysiąść na piętrach, jeśli każda z tych osób wysiada na czwartym, piątym lub szóstym piętrze? Pomijamy kolejność wysiadania z windy na poszczególnych piętrach.
Na parterze wieżowca mającego osiem pięter do windy wsiada pięć osób. Na ile sposobów te osoby mogą wysiąść na piętrach, jeśli żadne dwie osoby nie wysiadają na tym samym piętrze? Pomijamy kolejność wysiadania z windy na poszczególnych piętrach.
Na parterze wieżowca mającego osiem pięter do windy wsiada pięć osób. Na ile sposobów te osoby mogą wysiąść na piętrach, jeśli co najmniej jedna osoba wysiada na siódmym piętrze? Pomijamy kolejność wysiadania z windy na poszczególnych piętrach.
Każdemu spośród czterech uczniów przyporządkowujemy ocenę roczną z matematyki. Ile jest możliwych wyników tego przyporządkowania, jeżeli każdy z uczniów będzie miał inną ocenę?
Każdemu spośród czterech uczniów przyporządkowujemy ocenę roczną z matematyki. Ile jest możliwych wyników tego przyporządkowania, jeżeli każdy z uczniów może uzyskać dowolną ocenę?
Ile jest liczb dwucyfrowych, w których co najmniej jedna cyfra jest parzysta?
Ile jest liczb dwucyfrowych, w których co najmniej jedna cyfra jest nieparzysta?
Dany jest zbiór $X = \{1, 2, 3, 4, 5, 6, 7, 8, 9, 10\}$. Ile jest par uporządkowanych $(a, b)$ takich, że liczby $a, b$ należą do zbioru $X$ oraz liczba $a$ jest mniejsza od 3 lub liczba $b$ jest większa od 7?
Dany jest zbiór $X = \{1, 2, 3, 4, 5, 6, 7, 8, 9, 10\}$. Ile jest par uporządkowanych $(a, b)$ takich, że liczby $a, b$ należą do zbioru $X$ oraz liczba $a$ jest większa od $b$ lub liczba $b$ jest większa od liczby $9$?
Dany jest zbiór $X = \{1, 2, 3, 4, 5, 6, 7, 8, 9, 10\}$. Ile jest par uporządkowanych $(a, b)$ takich, że liczby $a, b$ należą do zbioru $X$ oraz liczba $a$ jest nie mniejsza niż 4 i jednocześnie liczba $b$ jest podzielna przez 3 lub przez 5?
Dany jest zbiór $X = \{1, 2, 3, 4, 5, 6, 7, 8, 9, 10\}$. Ile jest par uporządkowanych $(a, b)$ takich, że liczby $a, b$ należą do zbioru $X$ oraz liczba $a$ jest liczbą pierwszą lub liczba $b$ jest nie większa niż 6?
Z cyfr ze zbioru $X = \{0, 1, 2, 3\}$ tworzymy wszystkie liczby trzycyfrowe, przy czym cyfry w liczbie nie mogą się powtarzać. Narysuj drzewo, w którym gałęzie przedstawiają utworzone liczby trzycyfrowe. Ile jest wśród nich liczb podzielnych przez 3?
Oblicz, ile jest liczb trzycyfrowych, zbudowanych z cyfr należących do zbioru {1, 2, 3, 4, 5, 6, 7, 8, 9}, jeśli cyfry w liczbie nie mogą się powtarzać.
Pewna firma chce produkować ulotki, których każda ze stron ma mieć dwukolorowe tło: górna połowa danej strony ma mieć inny kolor, niż dolna połowa. Ile jest wzorów takich ulotek, jeśli firma ma do dyspozycji siedem kolorów?
Ile jest liczb trzycyfrowych o różnych cyfrach, utworzonych z cyfr należących do zbioru {1, 2, 3, 4, 5, 6, 7} i jednocześnie mniejszych od 444?
Ile jest liczb trzycyfrowych o różnych cyfrach, utworzonych z cyfr należących do zbioru {1, 2, 3, 4, 5, 6, 7, 8, 9} i jednocześnie mniejszych od 780?
Ile jest liczb trzycyfrowych o różnych cyfrach, podzielnych przez 25?
Ile jest liczb trzycyfrowych o różnych cyfrach, podzielnych przez 4?
Ile jest liczb czterocyfrowych o różnych cyfrach i jednocześnie podzielnych przez 25?
Ile jest liczb czterocyfrowych o różnych cyfrach i jednocześnie większych od 5238?
Ile jest liczb pięciocyfrowych o różnych cyfrach i jednocześnie podzielnych przez 4?
Ile jest liczb pięciocyfrowych o różnych cyfrach i jednocześnie większych od 60000?
Ile jest telefonicznych numerów komórkowych, składających się z dziewięciu cyfr takich, że pierwszą cyfrą jest 5 lub 6, trzecią cyfrą jest 0, a pozostałe cyfry nie są ani piątką, ani szóstką, ani zerem?
Ile jest telefonicznych numerów komórkowych, składających się z dziewięciu cyfr takich, że każda cyfra jest inna i na pierwszym miejscu nie występuje 0?
Ile jest telefonicznych numerów komórkowych, składających się z dziewięciu cyfr takich, że każda kolejna cyfra tego numeru jest liczbą o 1 mniejszą od poprzedniej?
Ile jest telefonicznych numerów komórkowych, składających się z dziewięciu cyfr takich, że pierwsza, trzecia, piąta, siódma i dziewiąta cyfra jest taka sama i jest liczbą nieparzystą, zaś pozostałe cyfry są różnymi liczbami parzystymi?
Na ile sposobów można ustawić w rzędzie 5 różnych książek?
W klasie jest 12 uczniów. Na ile sposobów można wybrać przewodniczącego i jego zastępcę, jeśli te funkcje nie mogą być pełnione przez tę samą osobę?
Z menu składającego się z 4 zup, 6 dań głównych i 3 deserów trzeba wybrać pełny obiad (jedna pozycja z każdej kategorii). Na ile sposobów można to zrobić?
Na ile sposobów można rozmieścić 8 różnych samochodów na parkingu mającym 12 miejsc parkingowych ustawionych w rzędzie, jeśli wszystkie samochody mają stać obok siebie?
Ile jest liczb czterocyfrowych, w których pierwsza cyfra jest większa od ostatniej, a cyfry mogą się powtarzać?
W turnieju szachowym bierze udział 16 graczy. Każdy gracz gra z każdym dokładnie jeden mecz. Ile jest możliwych wyników całego turnieju (każdy mecz może zakończyć się zwycięstwem jednego z graczy lub remisem)?
Na ile sposobów można podzielić grupę 10 osób na dwie drużyny po 5 osób każda, jeśli jedna drużyna gra w czerwonych koszulkach, a druga w niebieskich? Ile będzie takich podziałów, jeśli kolor koszulek nie ma znaczenia?
Ile jest sposobów pokolorowania sześcianu sześcioma różnymi kolorami (każda ściana innym kolorem), jeśli dwa kolorowania uważamy za takie same, gdy jeden można otrzymać z drugiego przez obrót sześcianu?
Na szachownicy $8 \times 8$ umieszczamy 8 wież tak, żeby żadne dwie nie znajdowały się w tej samej kolumnie ani w tym samym wierszu. Na ile sposobów można to zrobić, jeśli 3 określone wieże muszą znajdować się w pierwszym wierszu?
W grupie 20 osób każda zna co najmniej 3 inne osoby z grupy. Udowodnij, że można wybrać 4 osoby tak, żeby każda znała każdą z pozostałych trzech. Ile jest minimalnie różnych sposobów takiego wyboru w najgorszym przypadku?
W szkolnym konkursie matematycznym uczestniczy 15 uczniów: 8 dziewczynek i 7 chłopców. Na ile sposobów można wybrać 5-osobową drużynę, jeśli w drużynie musi być co najmniej 2 dziewczynek i co najmniej 2 chłopców?
Na szachownicy $4 \times 4$ chcemy umieścić 4 wieże tak, żeby się wzajemnie nie szachowały (żadne dwie nie mogą być w tym samym wierszu ani kolumnie). Dodatkowo, wieże muszą być umieszczone tak, żeby żadne dwie nie znajdowały się na głównej przekątnej szachownicy. Na ile sposobów można to zrobić?
Ile jest liczb sześciocyfrowych, w których cyfra setek jest większa od cyfry dziesiątek, cyfra dziesiątek jest większa od cyfry jedności, a cyfra setek tysięcy jest parzysta? Cyfry mogą się powtarzać, ale pierwsza cyfra nie może być zerem.
W bibliotece na półce stoi 12 książek: 5 z matematyki, 4 z fizyki i 3 z chemii. Na ile sposobów można ustawić te książki tak, żeby wszystkie książki z tego samego przedmiotu stały obok siebie, ale żadne dwie książki z matematyki nie stały bezpośrednio obok siebie?
Grupa 20 turystów chce się podzielić na 4 grupy zwiedzające różne muzea. Pierwsze muzeum może przyjąć maksymalnie 6 osób, drugie 7 osób, trzecie 4 osoby, a czwarte 5 osób. Na ile sposobów można podzielić turystów, jeśli każda grupa musi mieć co najmniej 2 osoby?
Hasło do sejfu składa się z 6 cyfr. Ile jest takich haseł, w których dokładnie 3 cyfry są parzyste, dokładnie 2 cyfry są nieparzyste większe od 5, a pozostałe cyfry są nieparzyste nie większe od 5? Cyfry w haśle mogą się powtarzać.
Na planszy do gry umieszczamy 8 pionków w dwóch rzędach po 4 pionki. W górnym rzędzie umieszczamy 4 pionki białe, a w dolnym 4 pionki czarne. Na ile sposobów możemy przestawić te pionki tak, żeby w każdym rzędzie znajdował się co najmniej jeden pionek każdego koloru?
Z cyfr 1, 2, 3, 4, 5, 6, 7 tworzymy wszystkie możliwe liczby czterocyfrowe bez powtórzeń. Ile spośród tych liczb jest podzielnych przez 4 i jednocześnie ma sumę cyfr podzielną przez 3?
W małym miasteczku jest 5 skrzyżowań połączonych drogami w kształcie pełnego grafu (każde skrzyżowanie połączone jest z każdym innym). Turysta chce przejść trasę odwiedzającą dokładnie 4 różne skrzyżowania, zaczynając i kończąc w tym samym punkcie. Ile jest takich tras, jeśli nie może przechodzić tą samą drogą dwukrotnie?
Firma organizuje losowanie nagród dla swoich 100 klientów. Przygotowano 10 nagród głównych, 20 nagród pocieszenia i 70 klientów nie otrzyma nagrody. Na ile sposobów można rozdać nagrody, jeśli wśród laureatów nagród głównych musi być dokładnie 3 klientów z grupy 25 klientów VIP, a wśród laureatów nagród pocieszenia co najmniej 8 klientów z tej grupy VIP?
